% Options for packages loaded elsewhere
\PassOptionsToPackage{unicode}{hyperref}
\PassOptionsToPackage{hyphens}{url}
%
\documentclass[
]{book}
\usepackage{amsmath,amssymb}
\usepackage{iftex}
\ifPDFTeX
  \usepackage[T1]{fontenc}
  \usepackage[utf8]{inputenc}
  \usepackage{textcomp} % provide euro and other symbols
\else % if luatex or xetex
  \usepackage{unicode-math} % this also loads fontspec
  \defaultfontfeatures{Scale=MatchLowercase}
  \defaultfontfeatures[\rmfamily]{Ligatures=TeX,Scale=1}
\fi
\usepackage{lmodern}
\ifPDFTeX\else
  % xetex/luatex font selection
\fi
% Use upquote if available, for straight quotes in verbatim environments
\IfFileExists{upquote.sty}{\usepackage{upquote}}{}
\IfFileExists{microtype.sty}{% use microtype if available
  \usepackage[]{microtype}
  \UseMicrotypeSet[protrusion]{basicmath} % disable protrusion for tt fonts
}{}
\makeatletter
\@ifundefined{KOMAClassName}{% if non-KOMA class
  \IfFileExists{parskip.sty}{%
    \usepackage{parskip}
  }{% else
    \setlength{\parindent}{0pt}
    \setlength{\parskip}{6pt plus 2pt minus 1pt}}
}{% if KOMA class
  \KOMAoptions{parskip=half}}
\makeatother
\usepackage{xcolor}
\usepackage{color}
\usepackage{fancyvrb}
\newcommand{\VerbBar}{|}
\newcommand{\VERB}{\Verb[commandchars=\\\{\}]}
\DefineVerbatimEnvironment{Highlighting}{Verbatim}{commandchars=\\\{\}}
% Add ',fontsize=\small' for more characters per line
\usepackage{framed}
\definecolor{shadecolor}{RGB}{248,248,248}
\newenvironment{Shaded}{\begin{snugshade}}{\end{snugshade}}
\newcommand{\AlertTok}[1]{\textcolor[rgb]{0.94,0.16,0.16}{#1}}
\newcommand{\AnnotationTok}[1]{\textcolor[rgb]{0.56,0.35,0.01}{\textbf{\textit{#1}}}}
\newcommand{\AttributeTok}[1]{\textcolor[rgb]{0.13,0.29,0.53}{#1}}
\newcommand{\BaseNTok}[1]{\textcolor[rgb]{0.00,0.00,0.81}{#1}}
\newcommand{\BuiltInTok}[1]{#1}
\newcommand{\CharTok}[1]{\textcolor[rgb]{0.31,0.60,0.02}{#1}}
\newcommand{\CommentTok}[1]{\textcolor[rgb]{0.56,0.35,0.01}{\textit{#1}}}
\newcommand{\CommentVarTok}[1]{\textcolor[rgb]{0.56,0.35,0.01}{\textbf{\textit{#1}}}}
\newcommand{\ConstantTok}[1]{\textcolor[rgb]{0.56,0.35,0.01}{#1}}
\newcommand{\ControlFlowTok}[1]{\textcolor[rgb]{0.13,0.29,0.53}{\textbf{#1}}}
\newcommand{\DataTypeTok}[1]{\textcolor[rgb]{0.13,0.29,0.53}{#1}}
\newcommand{\DecValTok}[1]{\textcolor[rgb]{0.00,0.00,0.81}{#1}}
\newcommand{\DocumentationTok}[1]{\textcolor[rgb]{0.56,0.35,0.01}{\textbf{\textit{#1}}}}
\newcommand{\ErrorTok}[1]{\textcolor[rgb]{0.64,0.00,0.00}{\textbf{#1}}}
\newcommand{\ExtensionTok}[1]{#1}
\newcommand{\FloatTok}[1]{\textcolor[rgb]{0.00,0.00,0.81}{#1}}
\newcommand{\FunctionTok}[1]{\textcolor[rgb]{0.13,0.29,0.53}{\textbf{#1}}}
\newcommand{\ImportTok}[1]{#1}
\newcommand{\InformationTok}[1]{\textcolor[rgb]{0.56,0.35,0.01}{\textbf{\textit{#1}}}}
\newcommand{\KeywordTok}[1]{\textcolor[rgb]{0.13,0.29,0.53}{\textbf{#1}}}
\newcommand{\NormalTok}[1]{#1}
\newcommand{\OperatorTok}[1]{\textcolor[rgb]{0.81,0.36,0.00}{\textbf{#1}}}
\newcommand{\OtherTok}[1]{\textcolor[rgb]{0.56,0.35,0.01}{#1}}
\newcommand{\PreprocessorTok}[1]{\textcolor[rgb]{0.56,0.35,0.01}{\textit{#1}}}
\newcommand{\RegionMarkerTok}[1]{#1}
\newcommand{\SpecialCharTok}[1]{\textcolor[rgb]{0.81,0.36,0.00}{\textbf{#1}}}
\newcommand{\SpecialStringTok}[1]{\textcolor[rgb]{0.31,0.60,0.02}{#1}}
\newcommand{\StringTok}[1]{\textcolor[rgb]{0.31,0.60,0.02}{#1}}
\newcommand{\VariableTok}[1]{\textcolor[rgb]{0.00,0.00,0.00}{#1}}
\newcommand{\VerbatimStringTok}[1]{\textcolor[rgb]{0.31,0.60,0.02}{#1}}
\newcommand{\WarningTok}[1]{\textcolor[rgb]{0.56,0.35,0.01}{\textbf{\textit{#1}}}}
\usepackage{longtable,booktabs,array}
\usepackage{calc} % for calculating minipage widths
% Correct order of tables after \paragraph or \subparagraph
\usepackage{etoolbox}
\makeatletter
\patchcmd\longtable{\par}{\if@noskipsec\mbox{}\fi\par}{}{}
\makeatother
% Allow footnotes in longtable head/foot
\IfFileExists{footnotehyper.sty}{\usepackage{footnotehyper}}{\usepackage{footnote}}
\makesavenoteenv{longtable}
\usepackage{graphicx}
\makeatletter
\def\maxwidth{\ifdim\Gin@nat@width>\linewidth\linewidth\else\Gin@nat@width\fi}
\def\maxheight{\ifdim\Gin@nat@height>\textheight\textheight\else\Gin@nat@height\fi}
\makeatother
% Scale images if necessary, so that they will not overflow the page
% margins by default, and it is still possible to overwrite the defaults
% using explicit options in \includegraphics[width, height, ...]{}
\setkeys{Gin}{width=\maxwidth,height=\maxheight,keepaspectratio}
% Set default figure placement to htbp
\makeatletter
\def\fps@figure{htbp}
\makeatother
\setlength{\emergencystretch}{3em} % prevent overfull lines
\providecommand{\tightlist}{%
  \setlength{\itemsep}{0pt}\setlength{\parskip}{0pt}}
\setcounter{secnumdepth}{5}
\usepackage{booktabs}
\ifLuaTeX
  \usepackage{selnolig}  % disable illegal ligatures
\fi
\usepackage[]{natbib}
\bibliographystyle{plainnat}
\usepackage{bookmark}
\IfFileExists{xurl.sty}{\usepackage{xurl}}{} % add URL line breaks if available
\urlstyle{same}
\hypersetup{
  pdftitle={Jacob McPherson, PharmD},
  pdfauthor={Jacob McPherson},
  hidelinks,
  pdfcreator={LaTeX via pandoc}}

\title{Jacob McPherson, PharmD}
\author{Jacob McPherson}
\date{2024-06-29}

\begin{document}
\maketitle

{
\setcounter{tocdepth}{1}
\tableofcontents
}
\chapter{About}\label{about}

This is a \emph{sample} book written in \textbf{Markdown}. You can use anything that Pandoc's Markdown supports; for example, a math equation \(a^2 + b^2 = c^2\).

\section{Usage}\label{usage}

Each \textbf{bookdown} chapter is an .Rmd file, and each .Rmd file can contain one (and only one) chapter. A chapter \emph{must} start with a first-level heading: \texttt{\#\ A\ good\ chapter}, and can contain one (and only one) first-level heading.

Use second-level and higher headings within chapters like: \texttt{\#\#\ A\ short\ section} or \texttt{\#\#\#\ An\ even\ shorter\ section}.

The \texttt{index.Rmd} file is required, and is also your first book chapter. It will be the homepage when you render the book.

\section{Render book}\label{render-book}

You can render the HTML version of this example book without changing anything:

\begin{enumerate}
\def\labelenumi{\arabic{enumi}.}
\item
  Find the \textbf{Build} pane in the RStudio IDE, and
\item
  Click on \textbf{Build Book}, then select your output format, or select ``All formats'' if you'd like to use multiple formats from the same book source files.
\end{enumerate}

Or build the book from the R console:

\begin{Shaded}
\begin{Highlighting}[]
\NormalTok{bookdown}\SpecialCharTok{::}\FunctionTok{render\_book}\NormalTok{()}
\end{Highlighting}
\end{Shaded}

To render this example to PDF as a \texttt{bookdown::pdf\_book}, you'll need to install XeLaTeX. You are recommended to install TinyTeX (which includes XeLaTeX): \url{https://yihui.org/tinytex/}.

\section{Preview book}\label{preview-book}

As you work, you may start a local server to live preview this HTML book. This preview will update as you edit the book when you save individual .Rmd files. You can start the server in a work session by using the RStudio add-in ``Preview book'', or from the R console:

\begin{Shaded}
\begin{Highlighting}[]
\NormalTok{bookdown}\SpecialCharTok{::}\FunctionTok{serve\_book}\NormalTok{()}
\end{Highlighting}
\end{Shaded}

\chapter{Statistics}\label{statistics}

\section{Introductory R}\label{introductory-r}

The official CRAN `*Intro2R'* \url{https://cran.r-project.org/doc/manuals/r-release/R-intro.html},

Wickham and Grolemund's *`R4DS'* \url{https://r4ds.had.co.nz/},

Douglas et al.'s `*Intro2R'* \url{https://intro2r.com/}

\section{Advanced R}\label{advanced-r}

Wickham's *`Advanced R'* \url{https://adv-r.hadley.nz/},

Wickham \& Bryan's *`R Packages'* \url{https://r-pkgs.org/},

Jeroen Janssens's *`DS at the CL'* \url{https://www.datascienceatthecommandline.com/1e/},

Other readings can include the RMarkdown and Bookdown readings:

Xie, Dervieux \& Riederer's *`R Markdown Cookbook'* \url{https://bookdown.org/yihui/rmarkdown-cookbook/} and Xie, Allaire \& Grolemund's R Markdown: `*The Definitive Guide'* \url{https://bookdown.org/yihui/rmarkdown/},

Xie's *`bookdown'* \url{https://bookdown.org/yihui/bookdown/} \& *`blogdown'* \url{https://bookdown.org/yihui/blogdown/}

Lovelace, Nowosad \& Muenchow's *`Geocomputation in R'* \url{https://geocompr.robinlovelace.net/},

Fay et al.'s *`Engineering Production-Grade Shiny Apps'* \url{https://engineering-shiny.org/}

\section{Introductory Statistical Programming}\label{introductory-statistical-programming}

I've found W. Chang's *`Cookbook for R'* \url{http://www.cookbook-r.com/}, UCLA's Intro to R \url{https://stats.oarc.ucla.edu/stat/data/intro_r/intro_r_interactive_flat.html} \& BU's Basic Statistical Analysis \url{https://sphweb.bumc.bu.edu/otlt/MPH-Modules/BS/R/R-Manual/R-Manual_print.html} the best single-page introduction for teaching R for statistics

Once you learn R and want a blend of R and statistical theory, A. Swoeney's \url{https://antoinesoetewey.com/} excellent \href{\%5Bhttps://github.com/AntoineSoetewey/statsandr}{`Stats and R' Blogdown}\url{https://github.com/AntoineSoetewey/statsandr} provides a PDF in `What statistical test should I do?' \url{https://statsandr.com/blog/files/overview-statistical-tests-statsandr.pdf} that users click the end-node links to follow.

More theory can be found at *Statistics for Biologists* \url{https://www.nature.com/collections/qghhqm} and its sub-page *Points of Significance* \url{https://www.nature.com/collections/qghhqm/pointsofsignificance}

*Handbook of Statistical Analyses Using R* HSAUR 3rd ed.~\url{https://rdrr.io/cran/HSAUR3/} entirely available online as individual chapter PDFs, with the associated HSAUR3 \url{https://cran.r-project.org/web/packages/HSAUR3/index.html} package in CRAN with Vignettes and official documentation reference manual \url{https://cran.r-project.org/web/packages/HSAUR3/HSAUR3.pdf}

\section{Intermediate Statistical Programming}\label{intermediate-statistical-programming}

Regression: I've struggled to find reputable open-sourced pages of regression education, let alone incorporation in R, but the PSU STAT 501 \url{https://online.stat.psu.edu/stat501/} has caught my attention

Ecology statistics with Oksanen's *vegan* GitHub \url{https://github.com/vegandevs/vegan}, rdocumentation \url{https://rdocumentation.org/packages/vegan} and CRAN \url{https://cran.r-project.org/web/packages/vegan/index.html}

\section{Advanced Statistical Programming}\label{advanced-statistical-programming}

High dimensional statistics can be learned from Borg \& Groenen's `*Modern Multidmensional Scaling'* \url{https://link.springer.com/book/10.1007/0-387-28981-X}

\section{Data Visualization}\label{data-visualization}

Data visualization should use ggplot2 from the Tidyverse \url{https://www.tidyverse.org/},

Wickham's `*ggplot2'* \url{https://ggplot2-book.org/},

Wilke's *`Fundamentals of Data Visualization'* \url{https://clauswilke.com/dataviz/},

W. Chang's *`R Graphics Cookbook'* 2e \url{https://r-graphics.org/}, and

**DEFINNITELY** give the ggplot2 extensions gallery \url{https://exts.ggplot2.tidyverse.org/gallery/} a peek, that I most highly recommend Patil's *ggstatplot* \url{https://github.com/IndrajeetPatil/ggstatsplot/}

\chapter{Genomics}\label{genomics}

\section{**Scientific Programming: Bioinformatics \& Computational Biology**}\label{scientific-programming-bioinformatics-computational-biology}

\subsection{Genomics}\label{genomics-1}

\href{\%5Bhttps://ncbiinsights.ncbi.nlm.nih.gov/}{National Center for Biotechnology and Information (NCBI)}{]}(\url{https://ncbiinsights.ncbi.nlm.nih.gov/}))

\href{\%5Bhttps://www.bv-brc.org/}{The Bacterial and Viral Bioinformatics Resource Center (BVBRC)}{]}(\url{https://www.bv-brc.org/}))

\href{\%5Bhttps://www.embl.org/}{European Molecular Biology Laboratories (EMBL)}{]}(\url{https://www.embl.org/})) \href{\%5Bhttps://www.ebi.ac.uk/research}{European Bioinformatics Institute (EBI)}{]}(\url{https://www.ebi.ac.uk/research}))

\href{\%5Bhttps://www.qiagen.com/us/knowledge-and-support/knowledge-hub}{QIAGEN's Knowledge Hub},{]}(\url{https://www.qiagen.com/us/knowledge-and-support/knowledge-hub}),) \href{\%5Bhttps://www.qiagen.com/us/knowledge-and-support/knowledge-hub/bench-guide}{Bench Guide}{]}(\url{https://www.qiagen.com/us/knowledge-and-support/knowledge-hub/bench-guide})) and \href{\%5Bhttps://digitalinsights.qiagen.com/}{Digital Insights},{]}(\url{https://digitalinsights.qiagen.com/}),)

\href{\%5Bhttps://www.sib.swiss/}{Swiss Institute for Bioinformatics (SIB)}{]}(\url{https://www.sib.swiss/})) which \href{\%5Bhttps://github.com/GeertvanGeest}{Geert van Geest}{]}(\url{https://github.com/GeertvanGeest})) introduced me to the SIB's \href{\%5Bhttps://github.com/sib-swiss/AWS-docker}{AWS-Docker}{]}(\url{https://github.com/sib-swiss/AWS-docker})) for getting RStudio Server, Jupyter and VSCode running on an AWS EC2 using Docker

\href{\%5Bhttps://www.thermofisher.com/us/en/home/technical-resources/learning-centers.html}{Thermo Fisher's Learning Centers}{]}(\url{https://www.thermofisher.com/us/en/home/technical-resources/learning-centers.html})) and

\href{\%5Bhttps://www.thermofisher.com/us/en/home/digital-science/thermo-fisher-connect.html}{Education Connect},{]}(\url{https://www.thermofisher.com/us/en/home/digital-science/thermo-fisher-connect.html}),)

\href{\%5Bhttps://www.illumina.com/science/education.html}{Illumina},{]}(\url{https://www.illumina.com/science/education.html}),)

The user manual for the \href{\%5Bhttps://resources.qiagenbioinformatics.com/manuals/clcmgm/300/index.php?manual=Create_K_mer_Tree.html}{k-mer trees}{]}(\url{https://resources.qiagenbioinformatics.com/manuals/clcmgm/300/index.php?manual=Create_K_mer_Tree.html})) and \href{\%5Bhttps://resources.qiagenbioinformatics.com/manuals/clcmgm/300/index.php?manual=Create_SNP_Tree.html}{SNP trees}{]}(\url{https://resources.qiagenbioinformatics.com/manuals/clcmgm/300/index.php?manual=Create_SNP_Tree.html})) are relatively more straight forward **WHEN USING WORKFLOWS**. Nonetheless, their visualization could use improvement; I naturally turn to **python** for bioinformatics and **R** for visualization

\subsection{Metagenomics}\label{metagenomics}

\href{\%5Bhttps://www.illumina.com/areas-of-interest/microbiology/microbial-sequencing-methods/16s-rrna-sequencing.html}{16s rRNA gene sequencing with Illumina},{]}(\url{https://www.illumina.com/areas-of-interest/microbiology/microbial-sequencing-methods/16s-rrna-sequencing.html}),) which feeds into either the current gold-standard open-source (python) tool \href{\%5Bhttps://qiime2.org/}{QIIME2}{]}(\url{https://qiime2.org/})) by \href{\%5Bhttps://www.nature.com/articles/s41587-019-0209-9}{Bolyen *et al.* 2019},{]}(\url{https://www.nature.com/articles/s41587-019-0209-9}),) the superseded (C++) gold-standard tool, \href{\%5Bhttps://github.com/mothur/mothur}{Mothur}{]}(\url{https://github.com/mothur/mothur})) by \href{\%5Bhttps://journals.asm.org/doi/10.1128/AEM.01541-09}{Schloss *et al.* 2009},{]}(\url{https://journals.asm.org/doi/10.1128/AEM.01541-09}),) which has a \href{\%5Bhttps://training.galaxyproject.org/archive/2021-10-01/topics/metagenomics/tutorials/mothur-miseq-sop/tutorial.html}{16S rRNA gene sequencing tutorial}{]}(\url{https://training.galaxyproject.org/archive/2021-10-01/topics/metagenomics/tutorials/mothur-miseq-sop/tutorial.html}))

\href{\%5Bhttps://resources.qiagenbioinformatics.com/manuals/clcmgm/300/index.php?manual=Introduction_Metagenomics.html}{16s rRNA gene sequencing with CLC}{]}(\url{https://resources.qiagenbioinformatics.com/manuals/clcmgm/300/index.php?manual=Introduction_Metagenomics.html})) with \href{\%5Bhttps://digitalinsights.qiagen.com/wp-content/uploads/2016/05/Characterizing-the-Microbiome-through-Targeted-Sequencing-of-Bacterial-16S-rRNA-and-Fungal-ITS-Regions_White-Paper_QIAGEN-Bioinformatics_0518_ww.pdf}{associated white paper}:{]}(\url{https://digitalinsights.qiagen.com/wp-content/uploads/2016/05/Characterizing-the-Microbiome-through-Targeted-Sequencing-of-Bacterial-16S-rRNA-and-Fungal-ITS-Regions_White-Paper_QIAGEN-Bioinformatics_0518_ww.pdf}):) The CLC workflow for 16S follows an \href{\%5Bhttps://resources.qiagenbioinformatics.com/manuals/clcmgm/300/index.php?manual=Amplicon_based_OTU_clustering.html}{amplicon-based OTU clustering workflow}{]}(\url{https://resources.qiagenbioinformatics.com/manuals/clcmgm/300/index.php?manual=Amplicon_based_OTU_clustering.html})) that uses \href{}{read trimming} using their \href{\%5Bhttps://resources.qiagenbioinformatics.com/manuals/clcassemblycell/400/index.php?manual=Quality_trimming.html}{`clc\_quality\_trim' program},{]}(\url{https://resources.qiagenbioinformatics.com/manuals/clcassemblycell/400/index.php?manual=Quality_trimming.html}),) but **I would rather use** \href{\%5Bhttp://www.usadellab.org/cms/?page=trimmomatic}{trimmomatic}{]}(\url{http://www.usadellab.org/cms/?page=trimmomatic})) by \href{\%5Bhttps://pubmed.ncbi.nlm.nih.gov/24695404/}{Bolger, Lihse \& Usadel, 2014};{]}(\url{https://pubmed.ncbi.nlm.nih.gov/24695404/});) \href{\%5Bhttps://resources.qiagenbioinformatics.com/manuals/clcmgm/300/index.php?manual=Filter_Samples_Based_on_Number_Reads.html}{filtering samples based on the number of reads};{]}(\url{https://resources.qiagenbioinformatics.com/manuals/clcmgm/300/index.php?manual=Filter_Samples_Based_on_Number_Reads.html});) *de novo* or reference-based {[}OTU clustering{]} \href{https://resources.qiagenbioinformatics.com/manuals/clcmgm/300/index.php?manual=OTU_clustering_parameters.html}{https://resources.qiagenbioinformatics.com/manuals/clcmgm/300/index.php?manual=OTU\_clustering\_parameters.html);};) \href{\%5Bhttps://resources.qiagenbioinformatics.com/manuals/clcmgm/300/index.php?manual=Remove_OTUs_with_Low_Abundance.html}{removal of low abundance OTUs};{]}(\url{https://resources.qiagenbioinformatics.com/manuals/clcmgm/300/index.php?manual=Remove_OTUs_with_Low_Abundance.html});) \href{\%5Bhttps://resources.qiagenbioinformatics.com/manuals/clcmgm/300/index.php?manual=Abundance_analysis.html}{OTU abundance analysis}{]}(\url{https://resources.qiagenbioinformatics.com/manuals/clcmgm/300/index.php?manual=Abundance_analysis.html})) **but I prefer R for this;**

\href{\%5Bhttps://resources.qiagenbioinformatics.com/manuals/clcmgm/300/index.php?manual=Align_OTUs_with_MUSCLE.html}{OTU nucleotide alignment with MUSCLE}{]}(\url{https://resources.qiagenbioinformatics.com/manuals/clcmgm/300/index.php?manual=Align_OTUs_with_MUSCLE.html})) by \href{\%5Bhttps://academic.oup.com/nar/article/32/5/1792/2380623?login=true}{Edgar, 2004}{]}(\url{https://academic.oup.com/nar/article/32/5/1792/2380623?login=true})) to \href{\%5Bhttp://resources.qiagenbioinformatics.com/manuals/clcgenomicsworkbench/current/index.php?manual}{generate a maximum likelihood phylogenetic tree},{]}(\url{http://resources.qiagenbioinformatics.com/manuals/clcgenomicsworkbench/current/index.php?manual}),) input for the \href{\%5Bhttps://resources.qiagenbioinformatics.com/manuals/clcmgm/300/index.php?manual=Estimate_Alpha_Beta_Diversities_workflow.html}{alpha- and beta-diversity workflow}{]}(\url{https://resources.qiagenbioinformatics.com/manuals/clcmgm/300/index.php?manual=Estimate_Alpha_Beta_Diversities_workflow.html})) **but I prefer \href{\%5Bhttps://vegandevs.github.io/vegan/index.html}{vegan}{]}(\url{https://vegandevs.github.io/vegan/index.html})) for this**

The microbial , \href{\%5Bhttps://github.com/picrust/picrust2}{PICRUST2},{]}(\url{https://github.com/picrust/picrust2}),) and the \href{\%5Bhttps://portal.hmpdacc.org/}{interactive Human Microbiome Project (iHMP)}{]}(\url{https://portal.hmpdacc.org/}))

\href{\%5Bhttps://www.illumina.com/areas-of-interest/microbiology/microbial-sequencing-methods/shotgun-metagenomic-sequencing.html}{Illumina SGS}{]}(\url{https://www.illumina.com/areas-of-interest/microbiology/microbial-sequencing-methods/shotgun-metagenomic-sequencing.html}))

In CLC, whole metagenome shotgun sequencing \href{\%5Bhttps://resources.qiagenbioinformatics.com/manuals/clcmgm/300/index.php?manual=Functional_analysis.html}{functional analysis}{]}(\url{https://resources.qiagenbioinformatics.com/manuals/clcmgm/300/index.php?manual=Functional_analysis.html})) first includes the user \href{\%5Bhttps://resources.qiagenbioinformatics.com/manuals/clcmgm/300/index.php?manual=De_Novo_Assemble_Metagenome.html\#sec:de_novo_assemble_metagenome}{*de novo* assembling a metagenome},{]}(\url{https://resources.qiagenbioinformatics.com/manuals/clcmgm/300/index.php?manual=De_Novo_Assemble_Metagenome.html\#sec:de_novo_assemble_metagenome}),) followed by annotation of the coding sequence (CDS) track with

\href{\%5Bhttps://resources.qiagenbioinformatics.com/manuals/clcmgm/300/index.php?manual=Annotate_CDS_with_Best_BLAST_Hit.html\#sec:annotate_cds_with_blast}{BLAST},{]}(\url{https://resources.qiagenbioinformatics.com/manuals/clcmgm/300/index.php?manual=Annotate_CDS_with_Best_BLAST_Hit.html\#sec:annotate_cds_with_blast}),) \href{\%5Bhttps://resources.qiagenbioinformatics.com/manuals/clcmgm/300/index.php?manual=Annotate_CDS_with_Pfam_Domains.html\#sec:annotate_cds_with_pfam}{Pfam domains},{]}(\url{https://resources.qiagenbioinformatics.com/manuals/clcmgm/300/index.php?manual=Annotate_CDS_with_Pfam_Domains.html\#sec:annotate_cds_with_pfam}),) and/or

\href{\%5Bhttps://resources.qiagenbioinformatics.com/manuals/clcmgm/300/index.php?manual=Download_GO_Database.html\#sec:download_go}{Gene Ontology (GO)}.{]}(\url{https://resources.qiagenbioinformatics.com/manuals/clcmgm/300/index.php?manual=Download_GO_Database.html\#sec:download_go}).) Then you map the original reads back to the annotated contigs using the `Map Reads to Reference' in the \href{\%5Bhttps://resources.qiagenbioinformatics.com/manuals/clcmgm/300/index.php?manual=Build_Functional_Profile.html\#sec:functional_profile}{Build Functional Profile}{]}(\url{https://resources.qiagenbioinformatics.com/manuals/clcmgm/300/index.php?manual=Build_Functional_Profile.html\#sec:functional_profile})) tool. The resulting output can be visualized using stacked bar charts and sunburst plots in \href{\%5Bhttps://resources.qiagenbioinformatics.com/manuals/clcmgm/300/index.php?manual=Visualization_OTU_abundance_tables.html\#sec:visualizationotu}{Visualization of the OTU abundance table}{]}(\url{https://resources.qiagenbioinformatics.com/manuals/clcmgm/300/index.php?manual=Visualization_OTU_abundance_tables.html\#sec:visualizationotu})) **but as you guessed, I prefer R for this.** As you might expect, **I might use** the open-source Linux OS (python) tool \href{\%5Bhttps://github.com/picrust/picrust2}{PICRUSt2}{]}(\url{https://github.com/picrust/picrust2})) by \href{\%5Bhttps://www.nature.com/articles/s41587-020-0548-6}{Douglas *et al.* 2020}{]}(\url{https://www.nature.com/articles/s41587-020-0548-6})) to do this too.

\chapter{Pharmacology}\label{pharmacology}

\section{Receptor Theory}\label{receptor-theory}

Terry Kenakin's `A Pharmacology Primer' \url{https://www.sciencedirect.com/book/9780128139578/a-pharmacology-primer}

\emph{Pharmacological Reviews} \url{https://pharmrev.aspetjournals.org/}

\emph{Nature Reviews Drug Discovery} \url{https://www.nature.com/nrd/}

\emph{British Journal of Pharmacology} (BPJ) \url{https://bpspubs.onlinelibrary.wiley.com/journal/14765381}

\emph{Journal of Pharmacology and Experimental Therapeutics} (JPET) \url{https://www.aspet.org/aspet/journals/the-journal-of-pharmacology-and-experimental-therapeutics}

\chapter{Favorite Journals}\label{favorite-journals}

**Academic Journals** starting with

\href{\%5Bhttps://www.scopus.com/}{SCOUPS}{]}(\url{https://www.scopus.com/})) that typically directs me to

\href{\%5Bhttps://www.nature.com/}{*Nature*},{]}(\url{https://www.nature.com/}),)

\href{\%5Bhttps://www.science.org/}{*Science*},{]}(\url{https://www.science.org/}),)

\href{\%5Bhttps://www.nature.com/nm/}{*Nature Medicine*},{]}(\url{https://www.nature.com/nm/}),)

\href{\%5Bhttps://www.science.org/journal/stm}{*Science Translational Medicine*}{]}(\url{https://www.science.org/journal/stm}))

**Laboratory methods** in

\href{\%5Bhttps://www.nature.com/nmeth/}{*Nature Methods*},{]}(\url{https://www.nature.com/nmeth/}),)

\href{\%5Bhttps://www.nature.com/nprot/}{*Nature Protocols*},{]}(\url{https://www.nature.com/nprot/}),) and

\href{\%5Bhttps://experiments.springernature.com/}{Springer\textbar Nature Experiments}{]}(\url{https://experiments.springernature.com/}))

**Cellular, Molecular Biology** in

\href{\%5Bhttps://brucealberts.ucsf.edu/current-projects/molecular-biology-of-the-cell/}{Bruce Albert's `Molecular Biology of the Cell'}{]}(\url{https://brucealberts.ucsf.edu/current-projects/molecular-biology-of-the-cell/}))

\href{\%5Bhttps://www.nature.com/nrm/}{*Nature Reviews Molecular Cell Biology*},{]}(\url{https://www.nature.com/nrm/}),)

\href{\%5Bhttps://www.cell.com/}{*Cell Press*},{]}(\url{https://www.cell.com/}),)

\href{\%5Bhttps://www.nature.com/nsmb/}{*Nature Structural \& Molecular Biology*},{]}(\url{https://www.nature.com/nsmb/}),)

\href{\%5Bhttps://www.nature.com/sigtrans/}{*Signal Transduction and Targeted Therapy*},{]}(\url{https://www.nature.com/sigtrans/}),) and

\href{\%5Bhttps://www.science.org/journal/signaling}{*Science Signaling*}{]}(\url{https://www.science.org/journal/signaling}))

**Clinical microbiology, metagenomics and microbial ecology (microbiome sciences)** in

\href{\%5Bhttps://www.amazon.com/Clinical-Microbiology-Twelfth-Michael-Pfaller/dp/1555819834}{Manual of Clinical Microbiology}{]}(\url{https://www.amazon.com/Clinical-Microbiology-Twelfth-Michael-Pfaller/dp/1555819834}))

\href{\%5Bhttps://www.amazon.com/Principles-Virology-Multi-ASM-Books/dp/1683670329/}{Principles of Virology}{]}(\url{https://www.amazon.com/Principles-Virology-Multi-ASM-Books/dp/1683670329/}))

\href{\%5Bhttps://www.nature.com/nrmicro/}{*Nature Reviews Microbiology*},{]}(\url{https://www.nature.com/nrmicro/}),)

\href{\%5Bhttps://journals.asm.org/journal/cmr}{*Clinical Microbiology Reviews*},{]}(\url{https://journals.asm.org/journal/cmr}),)

\href{\%5Bhttps://journals.asm.org/journal/jcm}{*Journal of Clinical Microbiology* (J Clin Micro)},{]}(\url{https://journals.asm.org/journal/jcm}),)

\href{\%5Bhttps://www.nature.com/nmicrobiol/}{*Nature Microbiology*},{]}(\url{https://www.nature.com/nmicrobiol/}),)

\href{\%5Bhttps://www.nature.com/ismej/}{*International Society for Microbial Ecology Journal* (ISMEJ)},{]}(\url{https://www.nature.com/ismej/}),)

\href{\%5Bhttps://www.cell.com/cell-host-microbe/}{*Cell Host \& Microbe*},{]}(\url{https://www.cell.com/cell-host-microbe/}),) the

\href{\%5Bhttps://www.nature.com/subjects/microbiology/nature}{*Nature* subject/Microbiology}{]}(\url{https://www.nature.com/subjects/microbiology/nature})) which includes the more narrow subjects

\href{\%5Bhttps://www.nature.com/subjects/microbiome/nature}{*Nature* subject/Microbiome},{]}(\url{https://www.nature.com/subjects/microbiome/nature}),)

\href{\%5Bhttps://www.nature.com/subjects/communities/nature}{*Nature* subject/Communities},{]}(\url{https://www.nature.com/subjects/communities/nature}),)

\href{\%5Bhttps://www.nature.com/subjects/metagenomics/nature}{*Nature* subject/Metagenomics},{]}(\url{https://www.nature.com/subjects/metagenomics/nature}),) and the ecology \& evolution journals

\href{\%5Bhttps://www.annualreviews.org/journal/ecolsys}{*Annual Review of Ecology, Evolution and Systematics*},{]}(\url{https://www.annualreviews.org/journal/ecolsys}),) and

\href{\%5Bhttps://www.nature.com/natecolevol/}{*Nature Ecology and Evolution*}{]}(\url{https://www.nature.com/natecolevol/}))

**Infectious Diseases and antimicrobial stewardship pharmacy practice** in

\href{\%5Bhttps://journals.asm.org/journal/aac}{*Antimicrobial Agents and Chemotherapy* (AAC)},{]}(\url{https://journals.asm.org/journal/aac}),)

\href{\%5Bhttps://www.nature.com/subjects/antimicrobials/nature}{*Nature* subject/Antimicrobials},{]}(\url{https://www.nature.com/subjects/antimicrobials/nature}),)

\href{\%5Bhttps://academic.oup.com/cid}{*Clinical Infectious Diseases* (CID)},{]}(\url{https://academic.oup.com/cid}),)

\href{\%5Bhttps://www.thelancet.com/journals/laninf/home}{*The Lancet Infectious Diseases*},{]}(\url{https://www.thelancet.com/journals/laninf/home}),)

\href{\%5Bhttps://www.nejm.org/infectious-disease}{*NEJM Infectious Diseases*},{]}(\url{https://www.nejm.org/infectious-disease}),) and the CDC's

\href{\%5Bhttps://www.cdc.gov/mmwr/}{*Morbidity and Mortality Weekly Report* (MMWR)}{]}(\url{https://www.cdc.gov/mmwr/})) and

\href{\%5Bhttps://wwwnc.cdc.gov/eid/}{*Emerging Infectious Diseases* (EID)}{]}(\url{https://wwwnc.cdc.gov/eid/}))

**Nucleic Acid Biochemistry**

\href{\%5Bhttps://www.amazon.com/Bioinformatics-Practical-Guide-Analysis-Proteins/dp/1119335582/}{Baxevanis, Bader \& Wishart's `Bioinformatics: A Practical Guide to the Analysis of Genes and Proteins'},{]}(\url{https://www.amazon.com/Bioinformatics-Practical-Guide-Analysis-Proteins/dp/1119335582/}),)

\href{\%5Bhttps://academic.oup.com/bioinformatics/}{*Bioinformatics*},{]}(\url{https://academic.oup.com/bioinformatics/}),)

\href{\%5Bhttps://www.nature.com/ng/}{*Nature Genetics*},{]}(\url{https://www.nature.com/ng/}),)

\href{\%5Bhttps://www.nature.com/nrg/}{*Nature Reviews Genetics*},{]}(\url{https://www.nature.com/nrg/}),)

\href{\%5Bhttps://genomebiology.biomedcentral.com/}{*Genome Biology*},{]}(\url{https://genomebiology.biomedcentral.com/}),)

\href{\%5Bhttps://academic.oup.com/nar/}{*Nucleic Acids Research*},{]}(\url{https://academic.oup.com/nar/}),) the

\href{\%5Bhttps://www.annualreviews.org/journal/genet}{*Annual Review of Genetics*},{]}(\url{https://www.annualreviews.org/journal/genet}),) and the

\href{\%5Bhttps://www.annualreviews.org/journal/genom}{*Annual Review of Genomics and Human Genetics*}{]}(\url{https://www.annualreviews.org/journal/genom}))

**Immunology** in

\href{\%5Bhttps://www.nature.com/ni/}{*Nature Immunology*},{]}(\url{https://www.nature.com/ni/}),)

\href{\%5Bhttps://www.nature.com/nri/}{*Nature Reviews Immunology*},{]}(\url{https://www.nature.com/nri/}),)

\href{\%5Bhttps://www.science.org/toc/sciimmunol/current}{*Science Immunology*},{]}(\url{https://www.science.org/toc/sciimmunol/current}),) and

\href{\%5Bhttps://www.nature.com/mi/}{*Mucosal Immunology*}{]}(\url{https://www.nature.com/mi/}))

**Artificial Intelligence** in

\href{\%5Bhttps://www.nature.com/natmachintell/}{*Nature Machine Intelligence*},{]}(\url{https://www.nature.com/natmachintell/}),)

\href{\%5Bhttps://www.science.org/journal/scirobotics}{*Science Robototics*},{]}(\url{https://www.science.org/journal/scirobotics}),) the

\href{\%5Bhttps://ieeexplore.ieee.org/xpl/RecentIssue.jsp?punumber=34}{*IEEE Transactions on Pattern Analysis and Machine Intelligence*},{]}(\url{https://ieeexplore.ieee.org/xpl/RecentIssue.jsp?punumber=34}),)

\href{\%5Bhttps://ieeexplore.ieee.org/xpl/RecentIssue.jsp?punumber=5962385}{*IEEE Transactions on Neural Networks and Learning Systems*},{]}(\url{https://ieeexplore.ieee.org/xpl/RecentIssue.jsp?punumber=5962385}),) the

\href{\%5Bhttps://onlinelibrary.wiley.com/journal/1098111x}{*International Journal of Intelligent Systems*}{]}(\url{https://onlinelibrary.wiley.com/journal/1098111x}))

\href{\%5Bhttps://www.journals.elsevier.com/information-sciences}{*Information Sciences*},{]}(\url{https://www.journals.elsevier.com/information-sciences}),) the

\href{\%5Bhttps://www.sciencedirect.com/journal/physics-of-life-reviews}{*Physics of Life Review*},{]}(\url{https://www.sciencedirect.com/journal/physics-of-life-reviews}),)

\href{\%5Bhttps://www.springer.com/journal/10462}{*Artificial Intelligence Review*},{]}(\url{https://www.springer.com/journal/10462}),)

\href{\%5Bhttps://www.journals.elsevier.com/knowledge-based-systems}{*Knowledge-Based Systems*},{]}(\url{https://www.journals.elsevier.com/knowledge-based-systems}),)

\href{\%5Bhttps://www.journals.elsevier.com/neural-networks}{*Neural Networks*},{]}(\url{https://www.journals.elsevier.com/neural-networks}),)

\href{\%5Bhttps://www.springer.com/journal/521}{*Neural Computing and Applications*},{]}(\url{https://www.springer.com/journal/521}),) the

\href{\%5Bhttps://www.springer.com/journal/11263}{*International Journal of Computer Vision*},{]}(\url{https://www.springer.com/journal/11263}),) and the journal

\href{\%5Bhttps://www.sciencedirect.com/journal/pattern-recognition}{*Pattern Recognition*}{]}(\url{https://www.sciencedirect.com/journal/pattern-recognition}))

\chapter{Artificial Intelligence}\label{artificial-intelligence}

**Artificial Intelligence** in

\href{\%5Bhttps://www.nature.com/natmachintell/}{*Nature Machine Intelligence*},{]}(\url{https://www.nature.com/natmachintell/}),)

\href{\%5Bhttps://www.science.org/journal/scirobotics}{*Science Robototics*},{]}(\url{https://www.science.org/journal/scirobotics}),) the

\href{\%5Bhttps://ieeexplore.ieee.org/xpl/RecentIssue.jsp?punumber=34}{*IEEE Transactions on Pattern Analysis and Machine Intelligence*},{]}(\url{https://ieeexplore.ieee.org/xpl/RecentIssue.jsp?punumber=34}),)

\href{\%5Bhttps://ieeexplore.ieee.org/xpl/RecentIssue.jsp?punumber=5962385}{*IEEE Transactions on Neural Networks and Learning Systems*},{]}(\url{https://ieeexplore.ieee.org/xpl/RecentIssue.jsp?punumber=5962385}),) the

\href{\%5Bhttps://onlinelibrary.wiley.com/journal/1098111x}{*International Journal of Intelligent Systems*}{]}(\url{https://onlinelibrary.wiley.com/journal/1098111x}))

\href{\%5Bhttps://www.journals.elsevier.com/information-sciences}{*Information Sciences*},{]}(\url{https://www.journals.elsevier.com/information-sciences}),) the

\href{\%5Bhttps://www.sciencedirect.com/journal/physics-of-life-reviews}{*Physics of Life Review*},{]}(\url{https://www.sciencedirect.com/journal/physics-of-life-reviews}),)

\href{\%5Bhttps://www.springer.com/journal/10462}{*Artificial Intelligence Review*},{]}(\url{https://www.springer.com/journal/10462}),)

\href{\%5Bhttps://www.journals.elsevier.com/knowledge-based-systems}{*Knowledge-Based Systems*},{]}(\url{https://www.journals.elsevier.com/knowledge-based-systems}),)

\href{\%5Bhttps://www.journals.elsevier.com/neural-networks}{*Neural Networks*},{]}(\url{https://www.journals.elsevier.com/neural-networks}),)

\href{\%5Bhttps://www.springer.com/journal/521}{*Neural Computing and Applications*},{]}(\url{https://www.springer.com/journal/521}),) the

\href{\%5Bhttps://www.springer.com/journal/11263}{*International Journal of Computer Vision*},{]}(\url{https://www.springer.com/journal/11263}),) and the journal

\href{\%5Bhttps://www.sciencedirect.com/journal/pattern-recognition}{*Pattern Recognition*}{]}(\url{https://www.sciencedirect.com/journal/pattern-recognition}))

\href{\%5Bhttps://github.com/deepmind}{DeepMind}{]}(\url{https://github.com/deepmind})) is an excellent gold-standard for the capability of deep-learning in the biological sciences, \href{\%5Bhttps://github.com/deepmind/alphafold}{AlphaFold}{]}(\url{https://github.com/deepmind/alphafold})) and the other \href{\%5Bhttps://github.com/deepmind/deepmind-research}{amazing discoveries}{]}(\url{https://github.com/deepmind/deepmind-research})) at DeepMind. The \href{\%5Bhttps://colab.research.google.com/github/deepmind/alphafold/blob/main/notebooks/AlphaFold.ipynb}{AlphaFold Colab}{]}(\url{https://colab.research.google.com/github/deepmind/alphafold/blob/main/notebooks/AlphaFold.ipynb})) is also freely available as a simplified implementation.

\href{\%5Bhttps://github.com/RosettaCommons/RoseTTAFold}{RoseTTAfold}{]}(\url{https://github.com/RosettaCommons/RoseTTAFold})) by \href{\%5Bhttps://www.science.org/doi/10.1126/science.abj8754}{Baek et al., 2021}{]}(\url{https://www.science.org/doi/10.1126/science.abj8754})) and \href{\%5Bhttps://twitter.com/peng_illinois/status/1538536909814874113}{OmegaFold}{]}(\url{https://twitter.com/peng_illinois/status/1538536909814874113}))

I think \href{\%5Bhttps://docs.aws.amazon.com/sagemaker/latest/dg/how-it-works-training.html}{AWS SageMaker}{]}(\url{https://docs.aws.amazon.com/sagemaker/latest/dg/how-it-works-training.html})) is best for its freedom of scalability; requires knowledge of \href{\%5Bhttps://github.com/data-science-on-aws/data-science-on-aws}{AWS Data Science},{]}(\url{https://github.com/data-science-on-aws/data-science-on-aws}),) and a review of the \href{\%5Bhttps://github.com/awslabs/amazon-sagemaker-workshop}{Sagemaker Workshop}{]}(\url{https://github.com/awslabs/amazon-sagemaker-workshop})) and \href{\%5Bhttps://github.com/aws/amazon-sagemaker-examples}{Examples}{]}(\url{https://github.com/aws/amazon-sagemaker-examples}))

The Microsoft \href{\%5Bhttps://github.com/py-why/dowhy}{DoWhy}{]}(\url{https://github.com/py-why/dowhy})) library for \href{\%5Bhttps://www.microsoft.com/en-us/research/blog/dowhy-a-library-for-causal-inference/}{causal inference}{]}(\url{https://www.microsoft.com/en-us/research/blog/dowhy-a-library-for-causal-inference/})) recently popped on my radar

Further reading can be found at

\href{\%5Bhttps://github.com/keras-team/keras}{Keras},{]}(\url{https://github.com/keras-team/keras}),)

\href{\%5Bhttps://github.com/tensorflow/tensorflow}{TensorFlow},{]}(\url{https://github.com/tensorflow/tensorflow}),)

\href{\%5Bhttps://github.com/scikit-learn/scikit-learn}{Scikit-Learn},{]}(\url{https://github.com/scikit-learn/scikit-learn}),)

\href{\%5Bhttps://www.statlearning.com/}{*`An Intro to Statistical Learning'*}{]}(\url{https://www.statlearning.com/}))

\href{\%5Bhttps://github.com/ageron/handson-ml3}{A. Geron's *Hands-On Machine-Learning*, 3rd ed.}{]}(\url{https://github.com/ageron/handson-ml3}))

\href{\%5Bhttps://github.com/fchollet/deep-learning-with-python-notebooks}{F. Chollet's *Deep-Learning with Python*}{]}(\url{https://github.com/fchollet/deep-learning-with-python-notebooks}))

\href{\%5Bhttps://github.com/NVIDIA/DeepLearningExamples}{Nvidia's `Deep-Learning Examples' in Python}{]}(\url{https://github.com/NVIDIA/DeepLearningExamples}))

\href{\%5Bhttps://machinelearningmastery.com/machine-learning-with-r/}{*Machine Learning with R*}{]}(\url{https://machinelearningmastery.com/machine-learning-with-r/}))

\href{\%5Bhttps://livebook.manning.com/book/deep-learning-with-r-second-edition/welcome/v-1/1}{*`Deep learning with R'* 2nd ed.}{]}(\url{https://livebook.manning.com/book/deep-learning-with-r-second-edition/welcome/v-1/1}))

\chapter{Computer Science}\label{computer-science}

\section{Computer science communities}\label{computer-science-communities}

\href{\%5Bhttps://www.r-project.org/}{R},{]}(\url{https://www.r-project.org/}),)

\href{\%5Bhttps://www.rstudio.com/}{RStudio},{]}(\url{https://www.rstudio.com/}),)

\href{\%5Bhttps://www.tidyverse.org/}{Tidyverse},{]}(\url{https://www.tidyverse.org/}),)

\href{\%5Bhttps://bookdown.org/}{Bookdown},{]}(\url{https://bookdown.org/}),)

\href{\%5Bhttps://www.python.org/}{Python},{]}(\url{https://www.python.org/}),)

\href{\%5Bhttps://biopython.org/}{BioPython},{]}(\url{https://biopython.org/}),)

\href{\%5Bhttps://www.anaconda.com/}{Anaconda}{]}(\url{https://www.anaconda.com/})) ,

\href{\%5Bhttps://julialang.org/}{Julia},{]}(\url{https://julialang.org/}),)

\href{\%5Bhttps://www.docker.com/}{Docker},{]}(\url{https://www.docker.com/}),)

\href{\%5Bhttps://git-scm.com/}{Git},{]}(\url{https://git-scm.com/}),)

\href{\%5Bhttps://github.com/features/copilot}{GitHub **Co-pilot**},{]}(\url{https://github.com/features/copilot}),)

\href{\%5Bhttps://www.latex-project.org/}{LaTeX},{]}(\url{https://www.latex-project.org/}),)

\href{\%5Bhttps://code.visualstudio.com/}{VSCode},{]}(\url{https://code.visualstudio.com/}),)

\href{\%5Bhttps://aws.amazon.com/}{AWS},{]}(\url{https://aws.amazon.com/}),)

\href{\%5Bhttps://www.ubuntu.com}{Ubuntu}{]}(\url{https://www.ubuntu.com}))

\section{Containerization Learning Resources}\label{containerization-learning-resources}

\href{\%5Bhttps://github.com/docker/labs}{The Official Docker Labs}{]}(\url{https://github.com/docker/labs})) by \href{\%5Bhttps://github.com/docker}{Docker},{]}(\url{https://github.com/docker}),)

\href{\%5Bhttps://dockerlabs.collabnix.com/}{The \#1 Docker Labs}{]}(\url{https://dockerlabs.collabnix.com/})) and \href{\%5Bhttps://collabnix.github.io/kubelabs/}{Kubernetes Labs}{]}(\url{https://collabnix.github.io/kubelabs/})) by \href{\%5Bhttps://github.com/collabnix/}{Collabnix},{]}(\url{https://github.com/collabnix/}),)

\href{\%5Bhttps://container.training/}{Containerization Training}{]}(\url{https://container.training/})) by \href{\%5Bhttps://github.com/jpetazzo/}{J. Petazzo},{]}(\url{https://github.com/jpetazzo/}),)

\href{\%5Bhttps://docker-curriculum.com/}{The Docker *for beginniners* Curriculum}{]}(\url{https://docker-curriculum.com/})) by \href{\%5Bhttps://github.com/prakhar1989/}{P. Srivastav}{]}(\url{https://github.com/prakhar1989/}))

\section{**Academic Programming**}\label{academic-programming}

Read in \href{\%5Bhttps://github.com/zotero/zotero}{Zotero}{]}(\url{https://github.com/zotero/zotero})) and cite using \href{\%5Bhttps://github.com/crsh/citr}{citr}{]}(\url{https://github.com/crsh/citr})) for RStudio; optional: VSCode using \href{\%5Bhttps://github.com/mblode/vscode-zotero}{Citation Picker}{]}(\url{https://github.com/mblode/vscode-zotero})) and \href{\%5Bhttps://github.com/retorquere/zotero-better-bibtex}{Better BibTeX}{]}(\url{https://github.com/retorquere/zotero-better-bibtex}))

**I made a public \href{\%5Bhttps://www.zotero.org/groups/4734738/jacobs_public_papers/library}{Zotero}{]}(\url{https://www.zotero.org/groups/4734738/jacobs_public_papers/library}))

for your reading enjoyment**

\chapter{Structural Biology}\label{structural-biology}

Structural biology is The Holy Grail of pharmacology; the dynamic states of pharmacological target activation/inactivation as determined by receptor theory

\section{Cryogenic electronic microscopy (CryoEM)}\label{cryogenic-electronic-microscopy-cryoem}

The \href{}{Theoretical and Computational Biophysics Group (TCBG)} provides \href{\%5Bhttps://www.ks.uiuc.edu/Training/Tutorials/}{tutorial-based training}{]}(\url{https://www.ks.uiuc.edu/Training/Tutorials/})) and a database of software for 3D molecular \href{\%5Bhttps://www.ks.uiuc.edu/Development/biosoftdb/biosoft.cgi?&category=3}{building},{]}(\url{https://www.ks.uiuc.edu/Development/biosoftdb/biosoft.cgi?&category=3}),) \href{\%5Bhttps://www.ks.uiuc.edu/Development/biosoftdb/biosoft.cgi?&category=2}{dynamics}{]}(\url{https://www.ks.uiuc.edu/Development/biosoftdb/biosoft.cgi?&category=2})) and \href{\%5Bhttps://www.ks.uiuc.edu/Development/biosoftdb/biosoft.cgi?&category=1}{visualization}{]}(\url{https://www.ks.uiuc.edu/Development/biosoftdb/biosoft.cgi?&category=1}))

Further resources can be found notably at the \href{\%5Bhttps://www.ebi.ac.uk/training/search-results?query=structural-biology}{EMBLI-EBI},{]}(\url{https://www.ebi.ac.uk/training/search-results?query=structural-biology}),) Stanford

The NIH established the following \href{\%5Bhttps://www.cryoemcenters.org/cryoem-centers/}{CryoEM Centers}{]}(\url{https://www.cryoemcenters.org/cryoem-centers/})) through \href{\%5Bhttps://www.nih.gov/news-events/news-releases/nih-funds-three-national-cryo-em-service-centers-training-new-microscopists}{awards}{]}(\url{https://www.nih.gov/news-events/news-releases/nih-funds-three-national-cryo-em-service-centers-training-new-microscopists})) to the

\href{\%5Bhttps://nccat.nysbc.org/}{National Center for CryoEM Access and Training (NCCAT)}{]}(\url{https://nccat.nysbc.org/})) at the \href{\%5Bhttps://nysbc.org/}{New York Structural Biology Center (NYSBC)},{]}(\url{https://nysbc.org/}),) the

\href{\%5Bhttps://pncc.labworks.org/}{Pacific Northwest Center for Cryo-EM}{]}(\url{https://pncc.labworks.org/})) at the \href{\%5Bhttps://www.pnnl.gov/}{Pacific Northwest National Laboratory(PNNL)},{]}(\url{https://www.pnnl.gov/}),) the

\href{\%5Bhttps://cryoem-s2c2.slac.stanford.edu/}{Stanford-SLACC Cryo-EM Center}{]}(\url{https://cryoem-s2c2.slac.stanford.edu/})) at the \href{\%5Bhttps://www6.slac.stanford.edu/}{National Accelerator Laboratory}{]}(\url{https://www6.slac.stanford.edu/}))

**Online learning resources include**:

\href{\%5Bhttps://em-learning.com/}{Thermo Fisher's EM-Learning}{]}(\url{https://em-learning.com/})) and

\href{\%5Bhttps://www.thermofisher.com/us/en/home/electron-microscopy/life-sciences/learning-center.html}{CryoEM Learning Center}{]}(\url{https://www.thermofisher.com/us/en/home/electron-microscopy/life-sciences/learning-center.html})) by

\href{\%5Bhttps://cryo-em-course.caltech.edu/}{Grant Jensen, CalTech},{]}(\url{https://cryo-em-course.caltech.edu/}),) and

\href{\%5Bhttps://research.pasteur.fr/en/team/nanoimaging/}{Matthijn Vos, the Pasteur Institute}{]}(\url{https://research.pasteur.fr/en/team/nanoimaging/}))

\href{\%5Bhttps://cryoem.yale.edu/cryo-em/workshops-and-online-courses}{Yale CryoEM},{]}(\url{https://cryoem.yale.edu/cryo-em/workshops-and-online-courses}),)

\href{\%5Bhttps://www.med.unc.edu/cryo-em/}{UNC CryoEM Core}{]}(\url{https://www.med.unc.edu/cryo-em/})) lists \href{\%5Bhttps://www.med.unc.edu/cryo-em/cryoem-links-and-resources/}{resources available},{]}(\url{https://www.med.unc.edu/cryo-em/cryoem-links-and-resources/}),) and

\href{\%5Bhttps://nramm.nysbc.org/workshops-and-courses/}{NIGMS National Resource for Automated Molecular Microscopy (NRAMM)}{]}(\url{https://nramm.nysbc.org/workshops-and-courses/}))

\section{X-Ray Crystallography}\label{x-ray-crystallography}

\chapter{References:}\label{references}

\section{Genomics}\label{genomics-2}

\href{https://doi.org/10.3390/jcm9010132}{Pereira, et al.~2020} `Bioinformatics and Computational Tools for Next-Generation Sequencing Analysis in Clinical Genetics'

\begin{longtable}[]{@{}
  >{\raggedright\arraybackslash}p{(\columnwidth - 4\tabcolsep) * \real{0.0178}}
  >{\raggedright\arraybackslash}p{(\columnwidth - 4\tabcolsep) * \real{0.5660}}
  >{\raggedright\arraybackslash}p{(\columnwidth - 4\tabcolsep) * \real{0.4151}}@{}}
\toprule\noalign{}
\begin{minipage}[b]{\linewidth}\raggedright
\end{minipage} & \begin{minipage}[b]{\linewidth}\raggedright
Wet-lab Protocols
\end{minipage} & \begin{minipage}[b]{\linewidth}\raggedright
Dry-Lab Protocols
\end{minipage} \\
\midrule\noalign{}
\endhead
\bottomrule\noalign{}
\endlastfoot
Thermo Fisher & \href{https://www.thermofisher.com/us/en/home/references/protocols.html}{Protocols} & \\
QIAGEN & \href{https://www.qiagen.com/us/knowledge-and-support/knowledge-hub/bench-guide}{Bench Guide} & \\
Illumina & \href{https://www.illumina.com/science/technology/next-generation-sequencing/beginners/tutorials.html}{NGS for Beginners}, \href{https://www.illumina.com/science/technology/next-generation-sequencing/beginners/glossary.html}{NGS Glossary}, \& \href{https://www.illumina.com/science/technology/next-generation-sequencing/beginners/ngs-workflow.html}{NGS Workflow Steps}

\href{https://support.illumina.com/bulletins/2016/05/dnarna-isolation-considerations-when-using-truseq-library-prep-kits.html}{DNA/RNA Isolation Considerations}, \href{https://support.illumina.com/bulletins/2016/05/library-quantification-and-quality-control-quick-reference-guide.html}{Library Quality Control}, \href{https://www.illumina.com/techniques/sequencing/ngs-library-prep.html}{NGS Library Preparation} \& \href{https://www.illumina.com/techniques/sequencing/ngs-library-prep/automation.html}{Library Prep Automation},

\href{https://www.illumina.com/systems/sequencing-platforms.html}{Sequencing Platforms}, \href{https://www.illumina.com/science/sequencing-method-explorer.html}{Sequencing Methods Explorer}, \href{https://www.illumina.com/science/technology/next-generation-sequencing/sequencing-technology.html}{Sequencing by Synthesis} & \href{https://www.illumina.com/informatics.html}{Bioinformatics},

\href{https://www.illumina.com/informatics/infrastructure-pipeline-setup/genomic-data-storage-security.html}{Cloud NGS Data}

\href{https://doi.org/10.1038/nmeth.2276}{Bokulich et al., Nature Methods 2013}

\href{https://en.wikipedia.org/wiki/FASTQ_format}{FastQ\_format}, \href{https://www.illumina.com/science/technology/next-generation-sequencing/plan-experiments/quality-scores.html}{Phred Scores}, \href{https://www.bioinformatics.babraham.ac.uk/projects/fastqc/}{FastQC} (\href{https://www.youtube.com/watch?v=GnWSXwQeJ_U}{NIAID YT} and \href{https://www.hadriengourle.com/tutorials/qc/}{Tutorial})

\href{https://github.com/timflutre/trimmomatic}{Trimmomatic} (\href{https://www.youtube.com/watch?v=xZbzPpW0NTk}{NIAID YT}) \\
\end{longtable}

\href{https://www.youtube.com/watch?v=U1KH9GCz4Xs&list=PLbVDKwGpb3XnwBACQ5w8L8NzjZ-6D0M8u}{Intro to Bioinformatics by QIIME2\_YT}

\href{https://www.youtube.com/watch?v=M2iXewkYHE0&list=PLbVDKwGpb3XmkQmoBy1wh3QfWlWdn_pTT}{Intro to QIIME2 Playlist} including \href{https://www.youtube.com/watch?v=QMqKd7HGBbQ&list=PLbVDKwGpb3XmkQmoBy1wh3QfWlWdn_pTT&index=9}{Importing \& Demultiplexing}, \href{https://www.youtube.com/watch?v=PmtqSa4Z1TQ&list=PLbVDKwGpb3XmkQmoBy1wh3QfWlWdn_pTT&index=11}{Denoising/Clustering} (\href{https://youtu.be/181bMy9YB9I?t=116}{DADA2} by \href{https://github.com/benjjneb/dada2}{The Callahan Lab}), \href{https://www.youtube.com/watch?v=Z9w2VZHJMZs&list=PLbVDKwGpb3XmkQmoBy1wh3QfWlWdn_pTT&index=15}{Taxonomic Classification}, \href{https://www.youtube.com/watch?v=g5BdGP4V5YA&list=PLbVDKwGpb3XmkQmoBy1wh3QfWlWdn_pTT&index=17}{Rarefaction} and \href{https://www.youtube.com/watch?v=0nEabGxFxlY&t=5s}{Phylogenetic}\href{https://www.youtube.com/watch?v=0nEabGxFxlY&list=PLbVDKwGpb3XmkQmoBy1wh3QfWlWdn_pTT&index=14}{Reconstruction}, \href{https://www.youtube.com/watch?v=tucm8y5xi88&list=PLbVDKwGpb3XmkQmoBy1wh3QfWlWdn_pTT&index=20}{Alpha}and \href{https://www.youtube.com/watch?v=7H_LhmGafOc&list=PLbVDKwGpb3XmkQmoBy1wh3QfWlWdn_pTT&index=22}{Beta-Diversity} and \href{https://www.youtube.com/watch?v=K3gzz5DGTc0&list=PLbVDKwGpb3XmkQmoBy1wh3QfWlWdn_pTT&index=28}{Longitudinal Studies} in \href{https://docs.qiime2.org/2022.2/}{Qiime2}

  \bibliography{book.bib,packages.bib}

\end{document}
